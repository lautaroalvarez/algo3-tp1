\subsection{C\'odigo Cruzando el puente}
	\begin{lstlisting}
package soluciones;

import java.util.LinkedList;
import java.util.Scanner;
import java.util.TreeSet;

import utils.Persona;

public class Ejercicio1{


	public static void main(String[] args){
		
		long time_start, time_end;
		time_start = System.currentTimeMillis();
		
		Ejercicio1 e = new Ejercicio1();
        
        
		Scanner capt = new Scanner(System.in);
        int estrategia = capt.nextInt();
        int N_arqs = capt.nextInt();
        int M_canis = capt.nextInt();
        capt.nextLine();
        LinkedList<Persona> inicial = new LinkedList<Persona>();
       
        
        while(N_arqs > 0){
        	Persona p = new Persona("A", capt.nextInt());
        	inicial.add(p);
        	N_arqs--;
        }

        capt.nextLine();
        while(M_canis > 0){
        	Persona p = new Persona("C", capt.nextInt());
        	inicial.add(p);
        	M_canis--;
        }
        capt.close();

        //int estrategia= 2;
        //System.out.println(e.Resolver(inicial,estrategia));
        int asd = e.Resolver(inicial,estrategia);
        //System.out.println("termine");
		time_end = System.currentTimeMillis();
		System.out.println(time_end - time_start);
	}
	
	@SuppressWarnings("unchecked")
	private int Resolver(LinkedList<Persona> iniciales, int tipo_estrategia){
		int tiempoMax = -1;/*Pensar una cota maxima para los iniciales*/

		int k = 0;
		for(Persona p : iniciales){	//Seteo las ids con numeros primos.
			p.setID(k);
			k++;					
		}
		
		Integer[] arr_instancia = new Integer[iniciales.size() + 1]; //Arreglo que indentifica la instancia actual personas mas linterna
		for(int i = 0; i < arr_instancia.length; i++){
			arr_instancia[i] = 1; //Estan todos en la izquierda, inclusive la linterna.
		}
		boolean linternaIzq = true;	//Solo por motivos declarativos
		LinkedList<Persona> vacia = new LinkedList<Persona>();
		LinkedList<Persona> izq = (LinkedList<Persona>) iniciales.clone();	//Clono para no arruinar la lista original.
		
		TreeSet<Integer> instancias = new TreeSet<Integer>();
		instancias.add(sacarID(arr_instancia));			//Agrego a la raiz como instancia.
		
		int res = -1;

		
		if(tipo_estrategia == 0) res = auxResolver0(izq, vacia, tiempoMax, 0, linternaIzq, instancias, arr_instancia);
		else if(tipo_estrategia == 1) res = auxResolver1(izq, vacia, tiempoMax, 0, linternaIzq, instancias, arr_instancia);
		else if(tipo_estrategia == 2) res = auxResolver2(izq, vacia, tiempoMax, 0, linternaIzq, instancias, arr_instancia);
		
		return res;
	}

@SuppressWarnings("unchecked")
private int auxResolver0(LinkedList<Persona> izq, LinkedList<Persona> der, int tiempoOptimo, int tiempoActual, boolean linternaIzq, TreeSet<Integer> instancias, Integer[] instancia_actual){
		
		/*CON LA ESTRATEGIA 0 SIEMPRE EMPIEZO MANDANDO UNO DE LOS DOS LADOS, Y LUEGO DOS*/
		
		if(izq.size() == 0) return minimo(tiempoOptimo, tiempoActual);	//Si la lista izquierda vino vacia es porque llegue a una solucion

		else if(maximo(tiempoOptimo, tiempoActual) == tiempoActual) return tiempoOptimo;	//Si tiempoActual es mas grande que el optimo no sigo
		
		
		
		boolean mandarUno = true; //Mando uno sii estoy en la derecha o en la izquierda pero con un solo tipo.
		boolean termine = false;

		int i = 0;
		int j = 1;
		Persona persona1;
		Persona persona2;

		int tMAX;
		do{
			
			Integer[] aux_instanciaActual = instancia_actual.clone();					
			
			
			LinkedList<Persona> auxIzq = (LinkedList<Persona>) izq.clone();
			LinkedList<Persona> auxDer = (LinkedList<Persona>) der.clone();

			if(linternaIzq){	//Si linterna esta en izquierda saco uno de la izquierda
				aux_instanciaActual[aux_instanciaActual.length-1] = 0;
				
				persona1 = izq.get(i);

				auxIzq.remove(i);
				aux_instanciaActual[persona1.dameID()] = 0;
				
				
				auxDer.add(persona1);
			}
			else { 	//Idem con derecha
				aux_instanciaActual[aux_instanciaActual.length-1] = 1;

				persona1 = der.get(i);

				auxDer.remove(i);
				aux_instanciaActual[persona1.dameID()] = 1;
				
				auxIzq.add(persona1);
			}

			tMAX = persona1.dameTiempo();

			if(mandarUno){	//Si tenia que mandar uno actualizo el indice y mas;
				i++;
				if(i == izq.size() && linternaIzq){
					mandarUno = false;
					i = 0;
					if(izq.size()==1) termine = true;
				}
				else if (i == der.size() && !linternaIzq){
					mandarUno = false;
					i = 0;
					if(der.size()==1) termine = true;
				}

			}

			else{		//Si tenia que mandar 2 tengo que sacar otro de donde este la linterna y actualizar los indices.
				if(linternaIzq) {
					persona2 = izq.get(j);
					auxIzq.remove(j-1);
					aux_instanciaActual[persona2.dameID()] = 0;

					
					auxDer.add(persona2);
					
				}
				else {
					persona2 = der.get(j);
					auxDer.remove(j-1);
					aux_instanciaActual[persona2.dameID()] = 1;
					
					auxIzq.add(persona2);
				}
				j++;

				if((linternaIzq && j == izq.size()) || (!linternaIzq && j == der.size())){	//Si j llego al final tengo que avanzar el i y seguir mandando.
					i++;
					j = i+1;
				}

				if(i==izq.size()-1 && linternaIzq){	//Si i llego al final es porque termine de mandar 2
					termine = true;

				}
				else if(i==der.size()-1 && !linternaIzq){
					termine = true;
				}

				tMAX = persona2.dameTiempo() > tMAX ? persona2.dameTiempo() : tMAX;

			}

			tiempoActual+= tMAX; //Sumo lo que cuesta este viaje.
			
			
			Integer instID = sacarID(aux_instanciaActual);			//calculo el ID de la nueva instancia				
			if(valido(auxIzq) && valido(auxDer) && !instancias.contains(instID)){	//Si la instancia nuva no es valida, o ya la repeti en esta rama, no sigo bajando.
			
				instancias.add(instID);	//Agrego la instancia "hijo" al conjunto 
				
				tiempoOptimo = auxResolver0( auxIzq, auxDer, tiempoOptimo, tiempoActual, !linternaIzq, instancias, aux_instanciaActual);
				
				instancias.remove(instID);	//Saco la instancia "hijo" anterior porque voy a bajar por otra rama.
			}
			tiempoActual-=tMAX;	//Arreglo el tiempoActual;

		}while(!termine);
		
		return tiempoOptimo;

	}


	@SuppressWarnings("unchecked")
	private int auxResolver1(LinkedList<Persona> izq, LinkedList<Persona> der, int tiempoOptimo, int tiempoActual, boolean linternaIzq, TreeSet<Integer> instancias, Integer[] instancia_actual){
		
		/*Con la ESTRATEGIA 1 siempre que estoy a la izquierda empiezo mandando de a dos y siempre que estoy a la derecha empiezo mandando de a uno*/
		
		if(izq.size() == 0) return minimo(tiempoOptimo, tiempoActual);	//Si la lista izquierda vino vacia es porque llegue a una solucion 

		else if(maximo(tiempoOptimo, tiempoActual) == tiempoActual) return tiempoOptimo;	//Si tiempoActual es mas grande que el optimo no sigo
		
		
		
		boolean mandarUno = !linternaIzq || (linternaIzq && izq.size() == 1); //Mando uno sii estoy en la derecha o en la izquierda pero con un solo tipo.
		boolean termine = false;

		int i = 0;
		int j = 1;
		Persona persona1;
		Persona persona2;

		int tMAX;
		do{
			
			Integer[] aux_instanciaActual = instancia_actual.clone();					
			
			
			LinkedList<Persona> auxIzq = (LinkedList<Persona>) izq.clone();
			LinkedList<Persona> auxDer = (LinkedList<Persona>) der.clone();

			if(linternaIzq){	//Si linterna esta en izquierda saco uno de la izquierda
				aux_instanciaActual[aux_instanciaActual.length-1] = 0;
				
				persona1 = izq.get(i);

				auxIzq.remove(i);
				aux_instanciaActual[persona1.dameID()] = 0;
				
				
				auxDer.add(persona1);
			}
			else { 	//Idem con derecha
				aux_instanciaActual[aux_instanciaActual.length-1] = 1;

				persona1 = der.get(i);

				auxDer.remove(i);
				aux_instanciaActual[persona1.dameID()] = 1;
				
				auxIzq.add(persona1);
			}

			tMAX = persona1.dameTiempo();

			if(mandarUno){	//Si tenia que mandar uno actualizo el indice y mas;
				i++;
				if(i == izq.size() && linternaIzq){
					termine = true;
				}
				else if (i == der.size() && !linternaIzq){
					mandarUno = false;
					i = 0;
					if(der.size()==1) termine = true;
				}

			}

			else{		//Si tenia que mandar 2 tengo que sacar otro de donde este la linterna y actualizar los indices.
				if(linternaIzq) {
					persona2 = izq.get(j);
					auxIzq.remove(j-1);
					aux_instanciaActual[persona2.dameID()] = 0;

					
					auxDer.add(persona2);
					
				}
				else {
					persona2 = der.get(j);
					auxDer.remove(j-1);
					aux_instanciaActual[persona2.dameID()] = 1;
					
					auxIzq.add(persona2);
				}
				j++;

				if((linternaIzq && j == izq.size()) || (!linternaIzq && j == der.size())){	//Si j llego al final tengo que avanzar el i y seguir mandando.
					i++;
					j = i+1;
				}

				if(i==izq.size()-1 && linternaIzq){	//Si i llego al final es porque termine de mandar 2
					mandarUno = true;
					i=0;
				}
				else if(i==der.size()-1 && !linternaIzq){
					termine = true;
				}

				tMAX = persona2.dameTiempo() > tMAX ? persona2.dameTiempo() : tMAX;

			}

			tiempoActual+= tMAX; //Sumo lo que cuesta este viaje.
			
			
			Integer instID = sacarID(aux_instanciaActual);			//calculo el ID de la nueva instancia				
			if(valido(auxIzq) && valido(auxDer) && !instancias.contains(instID)){	//Si la instancia nuva no es valida, o ya la repeti en esta rama, no sigo bajando.
			
				instancias.add(instID);	//Agrego la instancia "hijo" al conjunto 
				
				tiempoOptimo = auxResolver1( auxIzq, auxDer, tiempoOptimo, tiempoActual, !linternaIzq, instancias, aux_instanciaActual);
				
				instancias.remove(instID);	//Saco la instancia "hijo" anterior porque voy a bajar por otra rama.
			}
			tiempoActual-=tMAX;	//Arreglo el tiempoActual;

		}while(!termine);
		
		return tiempoOptimo;

	}

	
	@SuppressWarnings("unchecked")
	private int auxResolver2(LinkedList<Persona> izq, LinkedList<Persona> der, int tiempoOptimo, int tiempoActual, boolean linternaIzq, TreeSet<Integer> instancias, Integer[] instancia_actual){
		
		/* ESTRATEGIA 2 ES IGUAL A LA 1, SALVO QUE CUANDO MANDO UNA PERSONA, MANDO AL MINIMO */
		if(izq.size() == 0) return minimo(tiempoOptimo, tiempoActual);	//Si la lista izquierda vino vacia es porque llegue a una solucion 

		else if(maximo(tiempoOptimo, tiempoActual) == tiempoActual) return tiempoOptimo;	//Si tiempoActual es mas grande que el optimo no sigo
		
		
		
		boolean mandarUno = !linternaIzq || (linternaIzq && izq.size() == 1); //Mando uno sii estoy en la derecha o en la izquierda pero con un solo tipo.
		boolean termine = false;

		int tipo_mas_rapido = 0;
		int i = 0;
		int j = 1;
		Persona persona1;
		Persona persona2;

		int tMAX;
		do{
			
			Integer[] aux_instanciaActual = instancia_actual.clone();					
			
			
			LinkedList<Persona> auxIzq = (LinkedList<Persona>) izq.clone();
			LinkedList<Persona> auxDer = (LinkedList<Persona>) der.clone();

			if(linternaIzq){	//Si linterna esta en izquierda saco uno de la izquierda
				aux_instanciaActual[aux_instanciaActual.length-1] = 0;
				
				if(mandarUno){
					if(tipo_mas_rapido == 0) i = Dame_mas_veloz(izq, "canibal");
					else if(tipo_mas_rapido == 1) i = Dame_mas_veloz(izq, "arqueologo");
				}
				
				persona1 = izq.get(i);

				auxIzq.remove(i);
				aux_instanciaActual[persona1.dameID()] = 0;
				
				
				auxDer.add(persona1);
			}
			else { 	//Idem con derecha
				aux_instanciaActual[aux_instanciaActual.length-1] = 1;
				
				if(mandarUno){
					if(tipo_mas_rapido == 0) i = Dame_mas_veloz(der, "canibal");
					else if(tipo_mas_rapido == 1) i = Dame_mas_veloz(der, "arqueologo");
				}
				
				persona1 = der.get(i);

				auxDer.remove(i);
				aux_instanciaActual[persona1.dameID()] = 1;
				
				auxIzq.add(persona1);
			}

			tMAX = persona1.dameTiempo();

			if(mandarUno){	//Si tenia que mandar uno actualizo el indice y mas;
				tipo_mas_rapido++;
				if(tipo_mas_rapido == 2 && linternaIzq){
					termine = true;
				}
				else if (tipo_mas_rapido == 2 && !linternaIzq){
					mandarUno = false;
					i = 0;
					if(der.size()==1) termine = true;
				}

			}

			else{		//Si tenia que mandar 2 tengo que sacar otro de donde este la linterna y actualizar los indices.
				if(linternaIzq) {
					persona2 = izq.get(j);
					auxIzq.remove(j-1);
					aux_instanciaActual[persona2.dameID()] = 0;

					
					auxDer.add(persona2);
					
				}
				else {
					persona2 = der.get(j);
					auxDer.remove(j-1);
					aux_instanciaActual[persona2.dameID()] = 1;
					
					auxIzq.add(persona2);
				}
				j++;

				if((linternaIzq && j == izq.size()) || (!linternaIzq && j == der.size())){	//Si j llego al final tengo que avanzar el i y seguir mandando.
					i++;
					j = i+1;
				}

				if(i==izq.size()-1 && linternaIzq){	//Si i llego al final es porque termine de mandar 2
					mandarUno = true;
					i=0;
				}
				else if(i==der.size()-1 && !linternaIzq){
					termine = true;
				}

				tMAX = persona2.dameTiempo() > tMAX ? persona2.dameTiempo() : tMAX;

			}

			tiempoActual+= tMAX; //Sumo lo que cuesta este viaje.
			
			
			Integer instID = sacarID(aux_instanciaActual);			//calculo el ID de la nueva instancia				
			if(valido(auxIzq) && valido(auxDer) && !instancias.contains(instID)){	//Si la instancia nuva no es valida, o ya la repeti en esta rama, no sigo bajando.
			
				instancias.add(instID);	//Agrego la instancia "hijo" al conjunto 
				
				tiempoOptimo = auxResolver2( auxIzq, auxDer, tiempoOptimo, tiempoActual, !linternaIzq, instancias, aux_instanciaActual);
				
				instancias.remove(instID);	//Saco la instancia "hijo" anterior porque voy a bajar por otra rama.
			}
			tiempoActual-=tMAX;	//Arreglo el tiempoActual;

		}while(!termine);
		
		return tiempoOptimo;

	}
	
	private int Dame_mas_veloz(LinkedList<Persona> lista, String s) {
		int i = 0;
		Persona candidato = null;
		
		int k = 0;
		for(Persona p : lista){
			if (s == "canibal" && p.esCanibal()){
				if( candidato == null || candidato.dameTiempo() > p.dameTiempo() ){
					candidato = p;
					i = k;
				}
			}
			else if(s == "arqueologo" && !lista.get(k).esCanibal()){
				if( candidato == null || candidato.dameTiempo() > p.dameTiempo() ){
					candidato = p;
					i = k;
				}
			}
			k++;
		}
		
		
		return i;
	}

	private Integer sacarID(Integer[] arr_binario) {
		int acum = 0;
		int i = 0;
		for(Integer a : arr_binario){
			acum += a*Math.pow(2, i++);
		}
		return acum;
	}

	private boolean valido(LinkedList<Persona> grupo){
		int canibales = 0;
		int arqueologos = 0;
		for(Persona p : grupo ){
			if(p.esCanibal()) canibales++;
			else arqueologos++;
		}
		return (arqueologos >= canibales || canibales == grupo.size());
	}
	
	private int maximo(int a, int b){
		if(a== -1) return -1;
		else if(b==-1) return -1;
		else return Math.max(a,b);
	}
			
	private int minimo(int a, int b){
		if(maximo(a,b) == a) return b;
		else return a;
	}
			


}
	\end{lstlisting}
	
\subsection{C\'odigo Problemas en el camino}

	\begin{lstlisting}
    public static void Run(long p){
        //--Valor de la maximo exponente que puede llegar a tener un valor +1 o -1
        int maxExponente = (int) Math.ceil(Math.log(p)/Math.log(3));
        //--Lista que va a almacenar los exponentes que tengan asignadoun +1
        LinkedList<Long> listaPositivos = new LinkedList<Long>();
        //--Lista que va a almacenar los exponentes que tengan asignadoun -1
        LinkedList<Long> listaNegativos = new LinkedList<Long>();

        //--Llama a la funcion recursiva que resuelve el problema
        dameCombinacion(p, maxExponente, listaPositivos, listaNegativos);
        //---Llama a la funcion que imprime el resultado en el formato correcto
        imprimirSolucion(listaPositivos, listaNegativos);
    }
    private static void dameCombinacion(long p, int exponenteActual, LinkedList<Long> listaPositivos, LinkedList<Long> listaNegativos) {
        if (exponenteActual < 0) {
            return;
        }
        long potenciaActual = (long)Math.pow(3,exponenteActual);
        
        //--- Elije el valor para acercarse mas al numero p
        if (Math.abs(p - potenciaActual) < Math.ceil((double)potenciaActual / 2)) {
            //--- Le conviene poner +1
            listaPositivos.push(potenciaActual);
            p -= potenciaActual;
        } else if (Math.abs(p + potenciaActual) < Math.ceil((double)potenciaActual / 2)) {
            //--- Le conviene poner -1
            listaNegativos.push(potenciaActual);
            p += potenciaActual;
        }
        //--- Luego de restar/sumar la potencia actual
        //--- disminuye el coeficiente y se llama recursivamente
        dameCombinacion(p, exponenteActual - 1, listaPositivos, listaNegativos);
    }
	\end{lstlisting}

	
\subsection{C\'odigo Guardando el tesoro}
\subsubsection{Version Bottom Up}
	\begin{lstlisting}
package soluciones;

import java.util.*;

public class Ejercicio3_bu{

  public static class Tesoro{
    public int peso;
    public int valor;
    public int tipo;
    public Tesoro(int p, int v, int t){
      peso = p;
      valor = v;
      tipo = t;
    }
  }

  public static class Casilla{
    public int beneficio; // el beneficio maximo hasta ahi
    public int mochila; // mochila donde puse el objeto actual (si es que lo puse en alguno)
    public int hijoWeight1;
    public int hijoWeight2;
    public int hijoWeight3;
    public boolean visitado;

    public Casilla(int b){
      beneficio = b;
      mochila = -1;
      hijoWeight1 = -1;
      hijoWeight2 = -1;
      hijoWeight3 = -1;
      visitado = false;
    }

    public Casilla(int b, int m, int h1,int h2, int h3){
      beneficio = b;
      mochila = m;
      hijoWeight1 = h1;
      hijoWeight2 = h2;
      hijoWeight3 = h3;
      visitado = true;
    }
  }

  public static LinkedList<Tesoro> tesoros = new LinkedList<Tesoro>();
  public static Casilla[][][][] matriz;

  public static void main(String[] args){

    Scanner capt = new Scanner(System.in);
    int m = capt.nextInt();
    int n = capt.nextInt();
    int k[] = new int[3];
    for(int i = 0; i<m; ++i){
      k[i] = capt.nextInt();
    }
    for (int i = m; i<3; ++i){
      k[i] = 0;
    }

    for(int i = 0; i<n; ++i){
      int cantidad = capt.nextInt();
      int peso = capt.nextInt();
      int valor = capt.nextInt();
      int j = 0;
      while(j<cantidad){
        Tesoro nuevo = new Tesoro(peso, valor, i+1);
        Ejercicio3_bu.tesoros.add(nuevo);
        ++j;
      }
    }
    capt.close();
    Run(k,m);
  }

  public static void Run(int[] k, int m){
    Ejercicio3_bu.matriz = new Casilla[Ejercicio3_bu.tesoros.size()][k[0]+1][k[1]+1][k[2]+1];
 
    int weight[] = new int[3];
    for (int i = Ejercicio3_bu.tesoros.size()-1; i >= 0; i--){
      for (int j = 0; j<= k[0]; ++j){
        weight[0] = j;
        for (int q = 0; q <= k[1]; ++q){
          weight[1] = q;
          for (int w = 0; w <= k[2]; ++w){
            weight[2] = w;
            Ejercicio3_bu.matriz[i][j][q][w] = new Casilla(Ejercicio3_bu.tesoros.get(i).valor);
            knapsack(i, weight);
          }
        }
      }
    }

    System.out.println(Ejercicio3_bu.matriz[0][k[0]][k[1]][k[2]].beneficio);
    LinkedList<Integer>[] mochila = new LinkedList[m];
    for (int i = 0; i < m; i++) {
      mochila[i] = new LinkedList<Integer>();
    }

    for (int i = 0; i < Ejercicio3_bu.tesoros.size(); ++i) { // piso k porque ya no lo voy a usar
      Casilla actual = Ejercicio3_bu.matriz[i][k[0]][k[1]][k[2]];
      if (actual.mochila >= 0 && actual.mochila < 3) {
          mochila[actual.mochila].add(Ejercicio3_bu.tesoros.get(i).tipo);
      }
      k[0] = actual.hijoWeight1;
      k[1] = actual.hijoWeight2;
      k[2] = actual.hijoWeight3;
    }


    for (int i = 0; i < m; i++) {
      System.out.print(mochila[i].size());
      for (Integer a : mochila[i]) {
        System.out.print(" " + a);
      }
      System.out.println();
    }

  }

  public static int knapsack(int i, int weight[]){
    if (i >= Ejercicio3_bu.tesoros.size()) {
      // no hay mas objetos para recorrer
      return 0;
    }

    if (Ejercicio3_bu.matriz[i][weight[0]][weight[1]][weight[2]].visitado) {
      // ya calcule este casillero de mi Ejercicio3_bu.matriz
      return Ejercicio3_bu.matriz[i][weight[0]][weight[1]][weight[2]].beneficio;
    } else {
      // tengo que calcular esa posicion de la Ejercicio3_bu.matriz
      int pesoActual = Ejercicio3_bu.tesoros.get(i).peso;
      int valorActual = Ejercicio3_bu.tesoros.get(i).valor;

      int weightAux[][] = new int[4][3];
      for (int k = 0; k < 4; k++) {
        for (int j = 0; j < 3; j++) {
          weightAux[k][j] = weight[j];
          if (j==k) {
            weightAux[k][j] -= pesoActual;
          }
        }
      }

      int maximo = 3;
      int beneficio_maximo = knapsack(i+1, weightAux[maximo]);

      for (int j = 0; j < 3; j++) {
        if (pesoActual <= weight[j]) {
          int nuevoBeneficio = valorActual + knapsack(i+1, weightAux[j]);
          if (nuevoBeneficio > beneficio_maximo) {
            maximo = j;
            beneficio_maximo = nuevoBeneficio;
          }
        }
      }

      Ejercicio3_bu.matriz[i][weight[0]][weight[1]][weight[2]].beneficio = beneficio_maximo;
      Ejercicio3_bu.matriz[i][weight[0]][weight[1]][weight[2]].mochila = maximo;
      Ejercicio3_bu.matriz[i][weight[0]][weight[1]][weight[2]].hijoWeight1 = weightAux[maximo][0];
      Ejercicio3_bu.matriz[i][weight[0]][weight[1]][weight[2]].hijoWeight2 = weightAux[maximo][1];
      Ejercicio3_bu.matriz[i][weight[0]][weight[1]][weight[2]].hijoWeight3 = weightAux[maximo][2];
      Ejercicio3_bu.matriz[i][weight[0]][weight[1]][weight[2]].visitado = true;

      return beneficio_maximo;
    }
  }
}
	\end{lstlisting}
\subsubsection{Version Top Down}
	\begin{lstlisting}
package soluciones;

import java.util.*;

public class Ejercicio3_td{

  public static class Tesoro{
    public int peso;
    public int valor;
    public int tipo;
    public Tesoro(int p, int v, int t){
      peso = p;
      valor = v;
      tipo = t;
    }
  }

  public static class Casilla{
    public int beneficio; // el beneficio maximo hasta ahi
    public int mochila; // mochila donde puse el objeto actual (si es que lo puse en alguno)
    public int hijoWeight1;
    public int hijoWeight2;
    public int hijoWeight3;
    public boolean visitado;

    public Casilla(int b){
      beneficio = b;
      mochila = -1;
      hijoWeight1 = -1;
      hijoWeight2 = -1;
      hijoWeight3 = -1;
      visitado = false;
    }

    public Casilla(int b, int m, int h1,int h2, int h3){
      beneficio = b;
      mochila = m;
      hijoWeight1 = h1;
      hijoWeight2 = h2;
      hijoWeight3 = h3;
      visitado = true;
    }
  }


  public static void main(String[] args){
    Scanner capt = new Scanner(System.in);
    int m = capt.nextInt();
    int n = capt.nextInt();
    int k[] = new int[3];
    for(int i = 0; i<m; ++i){
      k[i] = capt.nextInt();
    }
    for (int i = m; i<3; ++i){
      k[i] = 0;
    }
    LinkedList<Tesoro> tesoros = new LinkedList<Tesoro>();

    for(int i = 0; i<n; ++i){
      int cantidad = capt.nextInt();
      int peso = capt.nextInt();
      int valor = capt.nextInt();
      int j = 0;
      while(j<cantidad){
        Tesoro nuevo = new Tesoro(peso, valor, i+1);
        tesoros.add(nuevo);
        ++j;
      }
    }
    capt.close();
    Run(k,m,tesoros);
  }

  public static void Run(int[] k, int m, LinkedList<Tesoro> tesoros){
    Casilla[][][][] matriz = new Casilla[tesoros.size()][k[0]+1][k[1]+1][k[2]+1];
    for (int i = 0; i<tesoros.size(); ++i){
      for (int j = 0; j<k[0]+1; ++j){
        for (int q = 0; q<k[1]+1; ++q){
          for (int w = 0; w<k[2]+1; ++w){
            matriz[i][j][q][w] = new Casilla(tesoros.get(i).valor);
          }
        }
      }
    }
    knapsack(matriz,tesoros,tesoros.size()-1, k[0], k[1], k[2]);
    System.out.println(matriz[tesoros.size()-1][k[0]][k[1]][k[2]].beneficio);
    LinkedList<Integer> mochila1 = new LinkedList<Integer>();
    LinkedList<Integer> mochila2 = new LinkedList<Integer>();
    LinkedList<Integer> mochila3 = new LinkedList<Integer>();
    for (int i = tesoros.size()-1; i>=0 ; --i){ // piso k porque no lo voy a volver a usar
      Casilla actual = matriz[i][k[0]][k[1]][k[2]];
      switch (actual.mochila){
        case 0:
          mochila1.add(tesoros.get(i).tipo);
          break;
        case 1:
          mochila2.add(tesoros.get(i).tipo);
          break;
        case 2:
          mochila3.add(tesoros.get(i).tipo);
          break;
      }
      k[0] = actual.hijoWeight1;
      k[1] = actual.hijoWeight2;
      k[2] = actual.hijoWeight3;
    }
    System.out.print(mochila1.size());
    for (Integer a : mochila1){
      System.out.print(" " + a);
    }
    System.out.println();
    if(m>=2){
      System.out.print(mochila2.size());
      for (Integer a : mochila2){
        System.out.print(" " + a);
      }
      System.out.println();
    }
    if(m>=3){
      System.out.print(mochila3.size());
      for (Integer a : mochila3){
        System.out.print(" " + a);
      }
      System.out.println();
    }
  }

  public static int knapsack(Casilla[][][][] matriz, LinkedList<Tesoro> tesoros, int i, int weight1, int weight2, int weight3){
    if(i<0){ // no hay mas objetos para recorrer
      return 0;
    }

    if (matriz[i][weight1][weight2][weight3].visitado){ // ya calcule este casillero de mi matriz
      return matriz[i][weight1][weight2][weight3].beneficio;
    }else{ // tengo que calcular esa posicion de la matriz
      int[] beneficios = new int[4];
      int pesoActual = tesoros.get(i).peso;
      int valorActual = tesoros.get(i).valor;
      if (pesoActual<= weight1) {
        beneficios[0] = valorActual + knapsack(matriz, tesoros, i-1, weight1-pesoActual, weight2, weight3);
      }
      if (pesoActual<= weight2) {
        beneficios[1] = valorActual + knapsack(matriz, tesoros, i-1, weight1, weight2-pesoActual, weight3);
      }
      if (pesoActual<= weight3) {
        beneficios[2] = valorActual + knapsack(matriz, tesoros, i-1, weight1, weight2, weight3-pesoActual);
      }
      beneficios[3] = knapsack(matriz, tesoros, i-1, weight1,weight2, weight3); // no lo puse

      int max = 0;
      int indMax = 0;
      for (int j = 0; j <4; ++j){
        if(max<= beneficios[j]){
          max = beneficios[j];
          indMax = j;
        }
      }	

      switch (indMax){ // indMax es la mochila en la que decidi poner el tesoro
        case 3:
          matriz[i][weight1][weight2][weight3] = new Casilla(beneficios[indMax], indMax, weight1, weight2, weight3);
          break;
        case 2:
          matriz[i][weight1][weight2][weight3] = new Casilla(beneficios[indMax], indMax, weight1, weight2, weight3-pesoActual);
          break;
        case 1:
          matriz[i][weight1][weight2][weight3] = new Casilla(beneficios[indMax], indMax, weight1, weight2-pesoActual, weight3);
          break;
        case 0:
          matriz[i][weight1][weight2][weight3] = new Casilla(beneficios[indMax], indMax, weight1-pesoActual, weight2, weight3);
          break;
      }
      return matriz[i][weight1][weight2][weight3].beneficio;
    }
  }
}
	\end{lstlisting}