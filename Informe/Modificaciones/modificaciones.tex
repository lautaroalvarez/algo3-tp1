\par En general, reevimos todo el informe y las implementaciones y modificamos las partes en las cuales recibimos correcciones y las partes que consideramos que debían mejorarse.

\par A continuación vamos a enunciar algunos cambios y correcciones del TP1 que creemos que son relevantes y es importante nombrarlas.

\subsection{Ejercicio 1: Cruzando el puente}

\begin{itemize}
	\item Se hicieron modificaciones generales en el informe. Se corrigieron errores mencionados en la corrección y se trató de ampliar las explicaciones detallando mejor los conceptos y ejemplificando de mejor manera.
	\item Se mejoró la explicación de los algoritmos, así como su pseudocódigo.
	\item Se implementaron podas se explicaron y se experimentó teniendo en cuenta estas estrategias.
\end{itemize}

\subsection{Ejercicio 2: Problemas en el camino}

\begin{itemize}
	\item Se corrigieron detalles de implementación.
	\item Se agregó la sección de experimentación, que era un faltante del TP1.
\end{itemize}

\subsection{Ejercicio 3: Guardando el tesoro}

\begin{itemize}
	\item Se mejoraron las explicaciones del informe.
	\item Se implementaron dos algoritmos, uno top down y otro bottom up. Esto por una corrección surgida en el coloquio del TP1.
	\item Se explicaron ambos algoritmos y se dieron las cotas de complejidad.
	\item Se experimentó comparando estas implementaciones y con distintos casos de entrada.
\end{itemize}