
A pesar de que en los casos aleatorios nuestras implementaciones dieron bastante similares vamos a proponer supuestos peores y mejores casos. Un mejor caso ser\'ia cuando el objeto es tan pesado que no llega a entrar en las mochilas mientras que un mejor caso ser\'ia cuando el objeto pesa 1 ya que siempre entra en las mochilas. En el primer caso en la implementaci\'on top down se ahorraria de calcular varias casillas de la matriz pero no en bootom up ya que calcula todos los casilleros. En el segundo caso estar\'iamos forzando a que calcule todas las casillas de la matriz y por esta raz\'on deber\'ia tenber un mayor tiempo de ejecuci\'on.

\begin{figure}[H]
	\centering
	\includegraphics[width=0.9\textwidth]{Problema3/img/casos.png}
	\caption{Resultados del experimento 5.}
	\label{fig: exp5_casos}
\end{figure}

En este gr\'afico se observa que los casos que tomamos no son peores ni mejores casos porque sus tiempos son muy parecidos.
